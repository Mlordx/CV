%% start of file `template.tex'.
%% Copyright 2006-2013 Xavier Danaux (xdanaux@gmail.com).
%
% This work may be distributed and/or modified under the
% conditions of the LaTeX Project Public License version 1.3c,
% available at http://www.latex-project.org/lppl/.


\documentclass[12pt,a4paper,sans]{moderncv}        % possible options include font size ('10pt', '11pt' and '12pt'), paper size ('a4paper', 'letterpaper', 'a5paper', 'legalpaper', 'executivepaper' and 'landscape') and font family ('sans' and 'roman')

% modern themes
\moderncvstyle{classic}                            % style options are 'casual' (default), 'classic', 'oldstyle' and 'banking'
\moderncvcolor{black}                                % color options 'blue' (default), 'orange', 'green', 'red', 'purple', 'grey' and 'black'
%\renewcommand{\familydefault}{\sfdefault}         % to set the default font; use '\sfdefault' for the default sans serif font, '\rmdefault' for the default roman one, or any tex font name
%\nopagenumbers{}                                  % uncomment to suppress automatic page numbering for CVs longer than one page

% character encoding
\usepackage[utf8]{inputenc}                       % if you are not using xelatex ou lualatex, replace by the encoding you are using
%\usepackage{CJKutf8}                              % if you need to use CJK to typeset your resume in Chinese, Japanese or Korean

\usepackage{xcolor}
\usepackage{amsfonts}

% adjust the page margins
\usepackage[scale=0.85]{geometry}
\setlength{\hintscolumnwidth}{2cm}                % if you want to change the width of the column with the dates
%\setlength{\makecvtitlenamewidth}{10cm}           % for the 'classic' style, if you want to force the width allocated to your name and avoid line breaks. be careful though, the length is normally calculated to avoid any overlap with your personal info; use this at your own typographical risks...

\usepackage{import}

% personal data
\name{Mateus}{Barros Rodrigues}
%\title{Resume}                               % optional, remove / comment the line if not wanted
%\address{Rua André Thomas, 661, Osasco - SP, Brazil, 06023-120}{}{}% optional, remove / comment the line if not wanted; the "postcode city" and and "country" arguments can be omitted or provided empty
\phone[mobile]{+55 11 97037-3867}                   % optional, remove / comment the line if not wanted
%\phone[fixed]{+55 11 2533-5147}                    % optional, remove / comment the line if not wanted
%\phone[fax]{+3~(456)~789~012}                      % optional, remove / comment the line if not wanted
\email{matbarrosrodrigues@gmail.com}                               % optional, remove / comment the line if not wanted
\homepage{github.com/mlordx}

% optional, remove / comment the line if not wanted
%\extrainfo{additional information}                 % optional, remove / comment the line if not wanted
%\photo[64pt][0.4pt]{picture}                       % optional, remove / comment the line if not wanted; '64pt' is the height the picture must be resized to, 0.4pt is the thickness of the frame around it (put it to 0pt for no frame) and 'picture' is the name of the picture file
%\quote{Some quote}                                 % optional, remove / comment the line if not wanted

% to show numerical labels in the bibliography (default is to show no labels); only useful if you make citations in your resume
%\makeatletter
%\renewcommand*{\bibliographyitemlabel}{\@biblabel{\arabic{enumiv}}}
%\makeatother
%\renewcommand*{\bibliographyitemlabel}{[\arabic{enumiv}]}% CONSIDER REPLACING THE ABOVE BY THIS

% bibliography with mutiple entries
%\usepackage{multibib}
%\newcites{book,misc}{{Books},{Others}}
%----------------------------------------------------------------------------------
%            content
%----------------------------------------------------------------------------------
\begin{document}
%\begin{CJK*}{UTF8}{gbsn}                          % to typeset your resume in Chinese using CJK
%-----       resume       ---------------------------------------------------------


\makecvtitle

\vspace{-40pt}

\section{Formação}
%\vspace{10pt}
\begin{itemize}
\item[]{\cventry{Fev 2017 - Jul 2019}{Mestrado em Ciência da Computação}{\newline Título: ``Single Source Shortest Paths in Simple Polygons''}{IME-USP}{\newline \textbf{Orientador:} Carlos Eduardo Ferreira}{
      \begin{minipage}{1.0\linewidth}
        \small  
        \begin{itemize}
        \item \textbf{Bolsa CAPES}
        \item \textbf{Monitorias Prestadas:} Introdução à Programação (2018/1) e Algoritmos e Estruturas de Dados II (2017/2).
        \end{itemize}
        % 
      \end{minipage}}}
    
\vspace{10pt}

\item[]{\cventry{Mar 2012 - Dez 2016}{Bacharelado em Ciência da Computação}{
    }{IME-USP}{\newline \textbf{Orientador:} Carlos Eduardo Ferreira}{
      \begin{minipage}{1.0\linewidth}
        \small   
        \begin{itemize}
        %\item \textbf{Description:} I studied and implemented a computational geometry algorithm to quickly find all
        %  line segments intersecting with a given rectangular window on $\mathbb{R}^2$ space and many related algorithms for point
        %  location and line segments intersection. More information at:
        %  \href{https://linux.ime.usp.br/~mlord/mac0499/}{{\color{blue}linux.ime.usp.br/\textasciitilde{}mlord/mac0499/}}.
          %
        \item \textbf{Monitorias Prestadas:} Introdução à Programação (2015/1) e Princípios de Desenvolvimento de Algoritmos (2015/2).
          %
        \end{itemize}
      \end{minipage}
    }}
%\vspace{3pt}
\end{itemize}

\section{Projetos}
% \vspace{10pt}
\begin{itemize}
\item[]{\cventry{Mar 2016 - Dez 2016}{Implementação de Algoritmo para Consultas de Segmentos em Janelas}{Trabalho de Conclusão de Curso}{IME-USP}{}{
      \begin{minipage}{1.0\linewidth}
        \small   
        \begin{itemize}
        \item Implementei e escrevi uma monografia sobre um algoritmo de geometria computacional que encontra todos os segmentos de reta que intersectam com uma dada janela retangular no espaço.
        \item Necessitou o estudo de diferentes estruturas de dados como a ``Segment Tree'' e ``Layered Range Tree'', além de algoritmos de localização de pontos.
        \item O código-fonte e a monografia estão disponíveis em:
          \href{https://www.linux.ime.usp.br/~mlord/mac0499}{\color{blue}linux.ime.usp.br/\textasciitilde{}mlord/mac0499.}
        \end{itemize}
      \end{minipage}
    }}

\vspace{5pt}
  
\item[]{\cventry{Mar 2015 - Jul 2015}{Sistema de Monitoria}{Contribuição no desenvolvimento da plataforma}{IME-USP}{}{
      \begin{minipage}{1.0\linewidth}
        \small   
        \begin{itemize}
        \item Fiz parte do desenvolvimento do site de administração das monitorias do IME-USP durante a disciplina ``Laboratório de Programação Extrema''.
        \item Implementação foi feita utilizando \textit{Ruby on Rails} e seguiu práticas de desenvolvimento ágil.
        \item Desenvolvi funcionalidades como a integração da plataforma com o sistema oficial da USP e a automatização de documentos relacionados à monitoria. Código disponível em:  \href{https://github.com/monitoria-imeusp/}{\color{blue}github.com/monitoria-imeusp.}
        \end{itemize}
      \end{minipage}
    }}
    
\vspace{5pt}
\item[]{\cventry{Ago 2014 - Dez 2014}{Algoritmo de 4-aproximação para a Decomposição Convexa Mínima de um Polígono Simples}{
      Implementação de algoritmo de geometria computacional}{IME-USP}{}{
      \begin{minipage}{1.0\linewidth}
        \small   
        \begin{itemize}
       % \item \textbf{Description:} Python implementation of Hertel and Mehlhorn's algorithm for convex decomposition of simple polygons for MAC0331 - Computational Geometry course. It required implementation of a Red-Black Binary tree, an algorithm for triangulation of monotone polygons and an implementation of the Winged Edge data structure. 
       %
        \item Implementação do algoritmo de Hertel e Mehlhorn feito durante a disciplina ``Geometria Computacional''.
        \item Necessitou o estudo de estruturas como ``Red-Black Binary Tree'' e  ``Doubly-connected Edge List'', além da implementação do algoritmo de triangulação de polígonos monótonos.
        \item  O código-fonte está disponível em: \href{https://github.com/mlordx/mac0331}{\color{blue}github.com/mlordx/mac0331}.
        \end{itemize}
      \end{minipage}
    }}
\end{itemize}

\section{Atividades Extra-Curriculares}
%\vspace{10pt}
\begin{itemize}

\item[]{\cventry{2014 - 2016}{Organização da Semana da Computação}{}{IME-USP}{}{
      \begin{minipage}{1.0\linewidth}
        \small   
        \begin{itemize}
        %\item \textbf{Description:} Helped surveying students' topics of interest, finding and inviting speakers and potential sponsors, arranging accommodation and transportations for speakers, among other event-managing related tasks. 
        %
        \item Foi um dos organizadores do evento.
        \item Auxiliou a definir os tópicos de interesse dos alunos, a conseguir acomodações e transporte para os palestrantes, além de outras tarefas administrativas do evento.
        \item Página do evento: \href{https://bcc.ime.usp.br/semana}{\color{blue}bcc.ime.usp.br/semana.}
        \end{itemize}
      \end{minipage}
    }}

\end{itemize}

\section{Outros}
\begin{itemize}
\item[]{\cventry{}{Idiomas}{}{}{}{
      \begin{minipage}{1.0\linewidth}
        \small   
        \begin{itemize}
        \item[.] Português: Nativo
        \item[.] Inglês: Fluente (Certificação TOEFL iBT - Nota 112/120)
        \end{itemize}
      \end{minipage}
}}
\vspace{5pt}

\item[]{\cventry{}{Linguagens de Programação}{}{}{}{
      \begin{minipage}{1.0\linewidth}
        \small   
        \begin{itemize}
        \item[.] Python
        \item[.] C++
        \item[.] C  
        \end{itemize}
      \end{minipage}
    }}
  
\end{itemize}

\end{document}
